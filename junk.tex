There are three definitions of optimality of hierarchical reinforcement learning:

\begin{definition}
    An \textbf{optimal} policy $\pi^*$ for MDP $M$ is a policy that achieves the highest cumulative reward
    among all policies for the MDP.
\end{definition}
\begin{definition}
    Given $Pi_H$, which is the set of all policies consistent with hierarchy $H$, 
    then an \textbf{hierarchical optimal} policy for MDP $M$ is a policy $\pi^* \in Pi_H$ that achieves the highest cumulative reward
    among all policies $\pi \in Pi_H$.
\end{definition}


Our goal is to find a method which 
In recursive optimality setting, subtasks are not aware of the context in which they are executed.
When the policy of parent subtask is suboptimal, subtasks pursue the suboptimal goal defined by the parent subtask.
Therefore, we cannot guarantee any optimality properties when some subtasks are suboptimal.

On the other hand, the subtasks in hierarchical optimality setting are aware of the context . 
They can pursue the correct goal regardless of the policy of parent subtask.
However, it is not good enough. Consider the task graph for the famous taxi problem in Fig. ?.  
Assume Root subtask is suboptimal and always choose Get subtask even when 
the passenger is in the taxi, Get subtask which is aware of current context knows
it should move to the destination of the passenger instead of the pickup location.
Nevertheless, it will never complete the task because it has no access to Putdown 
action. To handle this problem, we need to modify the hierarchy and let some subtasks 
to have access to the primitive actions which are essential for them to finish
the task on its own. For the taxi problem, we require that both Get and Put tasks


In Dietterich's work, all Max nodes are model-free. 
So let us proceed by changing "MaxRoot" node to the model-based one, and let "MaxExit"and "MaxGoal"
to be the regular model-free Max node.
The task for the $i$-th model-based Max node is to compute:
\begin{equation}
    Q^{\pi}(i, s, a) = Q_r^{\pi}(i, s, a) + Q_c^{\pi}(i, s, a).
    \label{eq:MaxQ}
\end{equation}
where the completion function $C^{\pi}(i, s, a)$ is the expected cumulative reward after
we complete (composite) action $a$ at state $s$ and before the termination of $i$-th Max node.
$V^{\pi}(a, s)$ is the 
expected cumulative reward when we execute action $a$ at state $s$.
The value $Q^{\pi}(i, s, a)$ is stored in the corresponding Q node.
For example, $Q^{\pi}(MaxRoot, s, GotoExit)$ is stored in "QExit" node.




The node queries its children node to get the value of $V^{\pi}(a, s)$.
It can be computed by:
\begin{equation}
    Q_r^{\pi}(i, s, j) = 
    \left\{\begin{array}{ll}
        Q_r^{\pi}(j, s, \pi_j(s)) + Q_c^{\pi}(j, s, \pi_j(s))& \mbox{if j is composite} \\
        \Sigma_{s'} P(s'|s, j)R(s'|s, j) & \mbox{if j is primitive} \\  
    \end{array} \right.
    \label{eq:Qr}
\end{equation}
In our example, to compute $Q^{\pi}(MaxRoot, s, GotoExit)$, "MaxRoot" node would query 
"MaxExit" node to get $V^{\pi}(GotoExit, s)$.

The $Q_c^{\pi}(i, s, a)$ can be computed by the model:
\begin{equation}
    Q_c^{\pi}(i, s, a) = \sum_{s', N} P_{s_i}^{\pi}(s', N|s, a)\gamma^N[Q_r^{\pi}(i, s', \pi_i(s')) + Q_c^{\pi}(i, s', \pi(s'))].
    \label{eq:Qc}
\end{equation}

\begin{equation}
    Q^{\pi}(i, s, a) = Q_r^{\pi}(i, s, a) + Q_c^{\pi}(i, s, a) + Q_e^{\pi}(i, s, a).
    \label{eq:HordQ}
\end{equation}

Note that $P(s'|s, \pi(s))$ and $V^{\pi}(a, s)$ are provided by the child Max nodes.
\begin{equation}
    Q_e^{\pi}(i, s, a) = \sum_{s', N} P_{x_i}(s', N|s, a)\gamma^N[Q^{\pi}(k, s', \pi_k(s'))],
    \label{eq:Qe}
\end{equation}
where $k$ is the index of subtask which executes $i$-th subtask.

Get subtask which is aware of current context knows
it should move to the destination of the passenger instead of the pickup location.
Nevertheless, it will never complete the task because it has no access to Putdown 
action. To handle this problem, we need to modify the hierarchy and let some subtasks 
to have access to the primitive actions which are essential for them to finish
the task on its own. For the taxi problem, we require that both Get and Put tasks
are able to solve the problem on their own.

Guestrin et al. \cite{ApproxFactor} proposed factored model-based approach for .

We begin by constructing a projection function $proj(s)$,
which projects state $s$ to an action-invariant space.
That is:
\begin{equation}
    if P(s'|s, a) > 0, proj(s') = proj(s), \forall a
\end{equation}
It means that the projected state does not change after the execution of 
any actions.
We also needs its dual function $\bar{proj}(s)$ to reconstruct 
the original state $s = \phi(proj(s), \bar{proj}(s))$.

Now we can model $P(s'|s, a)$ by:
\begin{equation}
    P(s'|s, a) = P(\bar{proj}(s')| \bar{proj}(s), proj(s), a)
\end{equation}

Note that we need to query $P(s'|s, a)$ from the child nodes with original representation.
Thus, $\phi(ps, \bar{ps})$ is adopted to compute the original $s'$.  

Although the state abstraction technique above provides us compact state representation, 
it doesn't change the size of the planning envelope. Thus we do not gain any computational
advantage by applying such an abstraction technique.

A key observation of this work is to adopt the approximated projection function which
does not project state $s$ to an action-invariant space.
The size of planning envelope grows exponentially with the number of features.
By assuming 
some features do not change during the execution of actions, we do gain computational advantage by
significantly reducing the size of the planning envelope. 
We lose the optimally with this approximation technique, but it is necessary because we want 
to apply our work beyond toy applications.
Our approach doesn't imply that
some features are completely ignored. The agent still computes the plan according to 
the full state information, but the transition model is simplified to only include the 
relevant information.

in which we can find some small set of variables which are more important than others. Take 
mobile robot navigation problem for example, 
what we concerned is the locaiton of the agent, 
for other slowing objects in surronding environment may not be a concerned 
Our contribution
We note that, however, for a stochastic environment, the size of planning envelope grows exponentially with the number of planning
steps. Predicting the next state is not enough since we may encounter different next states with the same state and action choice.
If we predict more than one states for each step, we will have a planning envelope which grows exponentially with the number of planning
steps. We need a way to restrict the size of planning envelope. Our idea is to separate the variables to planning variables 
and environment variables. We compute all of the possible values of planning variables, while assuming 
the environment variables to be static throughout the planning process.  

One of the difficulties to apply model-based RL to large problem 
Here we follow MAXQ approach and decompose Q-function as a combination of $Q_r$ and $Q_c$ terms.
It seems that we over-simplifies the problem. Nevertheless, there are many applications
in which we can find some small set of variables which are more important than others. Take 
mobile robot navigation problem for example, 
what we concerned is the locaiton of the agent, 
for other slowing objects in surronding environment may not be a concerned 
Our contribution



%Qr here
If 

%Qc here 
$Q_c^{\pi}(i, s, a)$ can be computed by solving equation \ref{eq:Qc}.
If action $a$ is a model-free subtask ($a \in C(H)$), 
we can estimated 
\begin{equation}
    \tilde{P}(x|s, a) = (1-\alpha)\tilde{P}(x|s, a) + \alpha [ \gamma^N \delta_{s'x}],
    \label{eq:approxP}
\end{equation}
where $s'$ is the observed state after composite action $a$ is executed at state $s$.
$k$ is the number of steps for action $a$ to finish. 
$x$ is the terminal state for the model-free Max node. $\delta_{s'x}=1$ if observed state $s'$
is terminal state $x$ and 0 otherwise.

For model-based child Max nodes, $P(s'|s, \pi(s))$ can be computed by equation \ref{eq:multiProb}.

$\alpha$ is the step-size parameter.
For model-free Max nodes, it can be estimated by \cite{option}:
The update of equation \ref{eq:approxP} is conducted after action $a$ terminates at some terminal state,
so we know the exact value of $k$. Note that all of terminal states of the Max node shall be updated after 
action $a$ finished.

We can also use equation \ref{eq:countP} to estimate $\tilde{P}(x|s, a)$, but the policy of the Max node
might change, so equation \ref{eq:approxP} is a better choice.
We follow k
We use  We need to learn 

There are three definitions of optimality of hierarchical reinforcement learning:

\begin{definition}
    \textbf{Optimality:} An optimal policy $\pi^*$ for MDP $M$ is a policy that achieves the highest cumulative reward
    among all policies for the MDP.
\end{definition}
\begin{definition}
    \textbf{Hierarchical Optimality:} Given $Pi_H$, which is the set of all policies consistent with hierarchy $H$, 
    then an hierarchical optimal policy for MDP $M$ is a policy $\pi^* \in Pi_H$ that achieves the highest cumulative reward
    among all policies $\pi \in Pi_H$.
\end{definition}
\begin{definition}
    Hierarchical Optimality Given $Pi_H$, which is the set of all policies consistent with hierarchy $H$, 
    then an hierarchical optimal policy for MDP $M$ is a policy $\pi^* \in Pi_H$ that achieves the highest cumulative reward
    among all policies $\pi \in Pi_H$.
\end{definition}

Our goal is to find a method which 
In recursive optimality setting, subtasks are not aware of the context in which they are executed.
When the policy of parent subtask is suboptimal, subtasks pursue the suboptimal goal defined by the parent subtask.
Therefore, we cannot guarantee any optimality properties when some subtasks are suboptimal.

On the other hand, the subtasks in hierarchical optimality setting are aware of the context . 
They can pursue the correct goal regardless of the policy of parent subtask.
However, it is not good enough. Consider the task graph for the famous taxi problem in Fig. ?.  
Assume Root subtask is suboptimal and always choose Get subtask even when 
the passenger is in the taxi, Get subtask which is aware of current context knows
it should move to the destination of the passenger instead of the pickup location.
Nevertheless, it will never complete the task because it has no access to Putdown 
action. To handle this problem, we need to modify the hierarchy and let some subtasks 
to have access to the primitive actions which are essential for them to finish
the task on its own. For the taxi problem, we require that both Get and Put tasks
are able to solve the problem on their own.

%---------------------------------------

Reinforcement learning methods can broadly be divided into two classes, model-based and model-free. Consider the problem illustrated in the
figure, of deciding which route to take on the way home from work on Friday evening. We can abstract this task as having states (in this case,
locations, notably of junctions), actions (e.g. going straight on or turning left or right at every intersection), probabilities of transitioning from one
state to another when a certain action is taken (these transitions are not necessarily deterministic, e.g. due to road works and bypasses), and
positive or negative outcomes (i.e. rewards or costs) at each transition from scenery, traffic jams, fuel consumed, etc. (which are again
probabilistic).
Model-based computation, illustrated in the left ‘thought bubble’, is akin to searching a mental map (a forward model of the task) that has been
learned based on previous experience. This forward model comprises knowledge of the characteristics of the task, notably, the probabilities of
different transitions and different immediate outcomes. Model-based action selection proceeds by searching the mental map to work out the longrun
value of each action at the current state in terms of the expected reward of the whole route home, and chooses the action that has the highest
value.
Model-free action selection, by contrast, is based on learning these long-run values of actions (or a preference order between actions) without
either building or searching through a model. RL provides a number of methods for doing this, in which learning is based on momentary
inconsistencies between successive estimates of these values along sample trajectories. These values, sometimes called cached values because
of the way they store experience, encompass all future probabilistic transitions and rewards in a single scalar number that denotes the overall future
worth of an action (or its attractiveness compared with other actions). For instance, as illustrated in the right ‘thought bubble’, experience may have
taught the commuter that on Friday evenings the best action at this intersection is to continue straight and avoid the freeway.
Model-free methods are clearly easier to use in terms of online decision-making; however, much trial-and-error experience is required to make the
values be good estimates of future consequences. Moreover, the cached values are inherently inflexible: although hearing about an unexpected
traffic jam on the radio can immediately affect action selection that is based on a forward model, the effect of the traffic jam on a cached propensity
such as ‘avoid the freeway on Friday evening’ cannot be calculated without further trial-and-error learning on days in which this traffic jam occurs.
Changes in the goal of behavior, as when moving to a new house, also expose the differences between the methods: whereas model-based
decision making can be immediately sensitive to such a goal-shift, cached values are again slow to change appropriately. Indeed, many of us have
experienced this directly in daily life after moving house. We clearly know the location of our new home, and can make our way to it by
concentrating on the new route; but we can occasionally take an habitual wrong turn toward the old address if our minds wander. Such
introspection, and a wealth of rigorous behavioral studies (see [15], for a review) suggests that the brain employs both model-free and model-based
decision-making strategies in parallel, with each dominating in different circumstances [14]. Indeed, somewhat different neural substrates underlie
each one [17].


We do not need to model all transition of all variables in the MDP.
We note that, however, for a stochastic environment, the size of planning envelope grows exponentially with the number of planning
steps. Predicting the next state is not enough since we may encounter different next states with the same state and action choice.
If we predict more than one states for each step, we will have a planning envelope which grows exponentially with the number of planning
steps. We need a way to restrict the size of planning envelope. Our idea is to separate the variables to planning variables 
and environment variables. We compute all of the possible values of planning variables, while assuming 
the environment variables to be static throughout the planning process.  
It seems that we over-simplifies the problem. Nevertheless, there are many applications
in which we can find some small set of variables which are more important than others. Take 
mobile robot navigation problem for example, 
what we concerned is the locaiton of the agent, 
for other slowing objects in surronding environment may not be a concerned 
Our contribution


There has been little previous work addressing planning
with linear function approximation in an online setting.

Our model seems over-simplified, however, there are some variables which are more 
infunential than others in many problems, it is reasonable to spend more resources
to model these variables. Take path for example.
Besides, a inevitable problem is the size of planning envelope. It 
We enumerate all possible states.
we separate the varible into planning variable and environment variables.
We only model the transition for planning varialbe and assuming environment variables are
static throughout the planning process.
One motivation application is the navigation problem. In navigation task,
what we concerned is the locaiton of the agent, 
for other slowing objects in surronding environment may not be a concerned 
since they do not have significant impact on our plan
Thus we can assign the location
as the planning variable and keep the rest of the information as environment variable.
With the approximation, we can significatly reduce the size of the planning envelope.
We don't have the trouble 
A separate decision tree is built to predict each
next state feature as well as the reward. Their algorithm
attempts to predict the absolute transition function, while
our approach is to predict the relative effects of transitions,
which may be easier in many domains. Degris et al.’s algo-
rithm calculates an ǫ-greedy policy through value iteration
based on the model provided by the decision trees.




Instead, we need a approximated model

There has been little previous work addressing planning
with linear function approximation in an online setting.
Paduraru (2007) treated this case, focusing mainly on sampling
stochastic models of a cascading linear form, but
also briefly discussing deterministic linear models. Degris,
Sigaud and Wuillemin (2006) developed a version of Dyna
based on approximations in the form of dynamic Bayes networks
and decision trees. Their system, SPITI, included
online learning and planning based on an incremental version
of structured value iteration (Boutilier, Dearden &
Goldszmidt 2000).




Reinforcement learning (RL) studies the problem of
finding effective solutions to sequential decision mak-
ing problems (Sutton & Barto, 1998). For many
agent-based applications, it is critical that an RL al-
gorithm be very sample efficient: that it takes very
few actions to learn an effective policy. We focus on
sample efficiency as the key evaluation criterion for
RL algorithms because in many agent-based applica-
tions, acquiring experiences can be very expensive and
time-consuming. Two of the main approaches towards
this goal are to incorporate generalization (function
approximation) into model-free methods and to de-
velop model-based algorithms. Model-based methods
achieve high sample efficiency by learning a model of
the domain and simulating experiences in their model,
thus saving precious samples in the real world.
This paper is motivated primarily by the observation
that the world is too large to explore exhaustively.
Consider, for example, the act of identifying your fa-
vorite restaurant. One possible strategy is to visit the
restaurants in order of their distances from your house,
visiting closest ones first. In most American cities, do-
ing so would likely lead you to visit several pizza par-
lors, fast food joints, and coffee shops before getting to
many gourmet establishments. Nonetheless, provided
that you have time to try every restaurant, eventually
you will find your favorite one.

The first of these advances is model-based RL. Early work in this direction
demonstrated that summarizing an agent’s experience into a model facilitates
the efficient reuse of data [1]. Later work investigated how the uncertainty in
the model can guide exploration, yielding the first (probabilistic) finite bounds
on the amount of data required to learn near-optimal behaviors [2,3]. Still, these
guarantees require that the agent exhaustively explore every state. Particularly
in large domains, this exploration can be impractical.
Third, hierarchical decomposition is perhaps the most intuitively appealing
extension of the standard approach, since we would like to imbue our learning
algorithms with the same awareness of structure that seems to allow us to
cope with the extraordinary complexity of the real world. Hierarchical RL has
been explored via work on temporal abstraction, in which temporally extended
abstract actions allow agents to reason above the level of primitive actions [6].
However, we still do not have a complete understanding of how hierarchy benefits
learning and therefore how to design or discover hierarchies.

Second, function approximation allows RL to cope with large or even infinite
state spaces by introducing generalization. It allows an algorithm to approximate
the long-term value of every action in every state using only a relatively
small set of parameters. Many state-of-the-art approaches employ model-free
algorithms and representations that attempt to estimate these values directly
from data [4,5], but they often st

Reinforcement learning studies the problem
of solving sequential decision making prob-
lems. Model-based methods learn an effec-
tive policy in few actions by learning a model
of the domain and simulating experience in
their models. Typical model-based methods
must visit each state at least once, which
can be infeasible in large domains. To over-

Each interaction provides additional
information that can be used to learn a better model of the
world’s dynamics, and because this change could result in a
different action being best (given the model), the planning
process should be repeated to take this into account. However,
planning is inherently a complex process; on large
problems it not possible to repeat it on every time step without
greatly slowing down the response time of the system.

\cite{approxModel}
There has been little previous work addressing planning
with linear function approximation in an online setting.
Paduraru (2007) treated this case, focusing mainly on sampling
stochastic models of a cascading linear form, but
also briefly discussing deterministic linear models. Degris,
Sigaud and Wuillemin (2006) developed a version of Dyna
based on approximations in the form of dynamic Bayes networks
and decision trees. Their system, SPITI, included
online learning and planning based on an incremental version
of structured value iteration (Boutilier, Dearden &
Goldszmidt 2000).

The factored Markov Decision Process (FMDP) is a formalism for sequential decision problems
that allows for the explicit representation of environmental structure not possible in the traditional
Markov Decision Process (MDP). Much work in the reinforcement learning literature has focused
on algorithms that exploit this structure to more efficiently learn or compute optimal solutions to
such problems [1, 2]. In some cases this has allowed for the application of reinforcement learning
techniques to environments much larger than could otherwise be handled feasibly by methods that
do not exploit structure. These approaches generally make use of structure by ignoring regions or
entire dimensions of the state space that are irrelevant to solving a particular task. Many of these
approaches make use of factored transition and reward models to represent these independencies.
The majority of work with FMDPs has focused on discrete environments; i.e., those with a finite
number of states and actions. Some of these approaches have presented methods for learning a
factored transition model online from experience, and for using that model to compute optimal
policies [3, 4, 5].

Each interaction provides additional
information that can be used to learn a better model of the
world’s dynamics, and because this change could result in a
different action being best (given the model), the planning
process should be repeated to take this into account. However,
planning is inherently a complex process; on large
problems it not possible to repeat it on every time step without
greatly slowing down the response time of the system.


Reinforcement learning studies the problem
of solving sequential decision making prob-
lems. Model-based methods learn an effec-
tive policy in few actions by learning a model
of the domain and simulating experience in
their models. Typical model-based methods
must visit each state at least once, which
can be infeasible in large domains. To over-
come this problem, the model learning algo-
rithm needs to generalize knowledge to un-
seen states and provide information about
the states in which it needs more experi-
ence. In this paper, we use existing super-
vised learning techniques to learn the model
of the domain. We empirically compare their
effectiveness at generalizing knowledge across
states on three different domains. Our results
indicate that tree-based models perform the
best after training on a small number of tran-
sitions, while support vector machines per-
form the best after a large number of transi-
tions.

%3.3 Local versus Global Optimality
%Using hierarchy reduces the size of the space that must be searched to find a good policy.
%However, a hierarchy constrains the space of possible policies so that it may not be
%possible to represent the optimal policy or its value function, and hence make it impossible
%to learn the optimal policy. If we cannot learn the optimal policy, the next best target
%would be to learn the best policy that is consistent with the given hierarchy. Two notions of
%optimality have been explored in the previous work on hierarchical reinforcement learning,
%hierarchical optimality and recursive optimality (Dietterich, 2000).
%Definition 3.3: A hierarchical optimal policy for MDP M is a hierarchical policy which
%has the best performance among all policies consistent with the given hierarchy. In other
%words, hierarchical optimality is a global optimum consistent with the given hierarchy. In
%this form of optimality, the policy for each individual subtask is not necessarily optimal,
%but the policy for the entire hierarchy is optimal. The HAMQ HRL algorithm (Parr, 1998)
%and the SMDP Q-learning algorithm for a fixed set of options (Sutton et al., 1999; Precup,
%2000) both converge to a hierarchically optimal policy. 
%Definition 3.4: Recursive optimality, first introduced by Dietterich (2000), is a weaker
%but more flexible form of optimality which only guarantees that the policy of each subtask
%is optimal given the policies of its children. It is an important and flexible form of
%optimality because it permits each subtask to learn a locally optimal policy while ignoring
%the behavior of its ancestors in the hierarchy. This increases the opportunity for subtask
%47
%sharing and state abstraction. The MAXQ-Q HRL algorithm (Dietterich, 2000) converges
%to a recursively optimal policy.

%\begin{definition}
    %$\mathbb{C}(H) = \{N_1, N_2, \dots, N_k\}$ is a leaf cover of MaxQ hierarchy $H$ if 
    %we can divide H into several disjoint partitions $P = \{P_1, \dots P_m\}$ after removing all nodes in $\mathbb{C}(H)$.
    %And the root node does not share the partition with any leaf nodes. 
    %$\exists N_i \in \mathbb{C(H)} \suchthat N_i$ is an ancestor of $l$.
%\end{definition}

%\begin{definition}
    %A subtask policy $\pi_i$ is optimal if $\pi_i^{hg*}(s)=\pi_p^*(s) \forall s \in S_i$,
    %where $\pi_p^*(s)$ is the flat optimal policy.
%\end{definition}


%\begin{theorem}
    %If the corresponding subtask policy $\pi_i$ is optimal $\forall N_i \in \mathbb{C}(H)$,
    %the policy $\pi_j$ of the ancestor of $N_i$ is also optimal for all possible $\pi_j$.
%\end{theorem}
%Proof: $\forall N_j$: the ancestor of some node $\in \mathbb{C}(H)$, policy $\pi_j(s)$ would invoke some 
%node $N_i$ and execute its policy $\pi_i^{hg*}(s)$. Since the policy is optimal, $\pi_j^{hg*}(s)=\pi_p^*(s)$.

%\begin{center}
%\begin{tabular}{@{}lp{6cm}@{}}
%\hline
%Algorithm: ExecuteHGPolicy\\
%\hline
%Initialize $\hat{Q_0}$ arbitrarily\\
%$t \leftarrow 0$\\
%Repeat (for each episode)\\
%\ \ \ \ \ \ Initialize $s$\\
%\ \ \ \ \ \ Choose $a$ based on $s$ using policy derived from $\hat{Q_t}$ (e.g., $\epsilon$-greedy method)\\
%\ \ \ \ \ \ Repeat (for each step of episode):\\
%\ \ \ \ \ \ \ \ \ \ \ \ Take action $a$, obtain reward $r$ and next state $s'$ from the environment\\
%\ \ \ \ \ \ \ \ \ \ \ \ Choose $a'$ based on $s'$ using policy derived from $\hat{Q_t}$ (e.g., $\epsilon$-greedy method)\\
%\ \ \ \ \ \ \ \ \ \ \ \ $\hat{q_t} \leftarrow \hat{Q_t}(s, a) + \alpha [r + \gamma \hat{Q_t}(s', a')-\hat{Q_t}(s, a)]$ \\
%\ \ \ \ \ \ \ \ \ \ \ \ Update $\hat{Q_t}$ with $(s, a, \hat{q_t})$ by online L1 regression to produce $\hat{Q_{t+1}}$\\
%\ \ \ \ \ \ \ \ \ \ \ \ $s \leftarrow s'$\\
%\ \ \ \ \ \ \ \ \ \ \ \ $a \leftarrow a'$\\
%\ \ \ \ \ \ \ \ \ \ \ \ $t \leftarrow t+1$\\
%\ \ \ \ \ \ Until $s$ is terminal\\
%\hline  
%\end{tabular}
%\end{center}

%Note that Q function is:
%\begin{align}
    %Q^{\pi}(s, \pi(s)) &= Q^{\pi}(i, s, \pi(s)) + Q_E^{\pi}(i, s, \pi(s))\\
    %&= V^{\pi}(\pi(s), s) + C^{\pi}(i, s, \pi(s)) + Q_E^{\pi}(i, s, \pi(s))
%\end{align}

%Andre and Russell then prove:
%\begin{theorem}
    %If $V^{*}, C^*, Q_E^*$ are solutions to equation \ref{eq:V}, \ref{eq:C},
    %and \ref{eq:QE} for $\pi^*$, then $Q^* = V^{*} + C^* + Q_E^*$ is a solution
    %to the standard Bellman equation.
%\end{theorem}
%\begin{theorem}
    %Decomposed value iteration and policy iteration algorithms derived from equation
    %\ref{eq:V}, \ref{eq:C}, and \ref{eq:QE} converge to $V^{*}, C^*, Q_E^*$, and $\pi^*$.
%\end{theorem}

%However, we need something stronger:
%\begin{theorem}
    %Let $\mathbb\{C\} = \{N_1, N_2, \dots\}$. If 
    %(a) $\forall N_i \in \mathbb{C}$, the descendants of Max node $N_i$ covers all primitive Max nodes
    %and (b) for all path from root node to leaf node must pass at least one of the nodes in $\mathbb{C}$,
    %then the agent has hierarchical optimal policy $\pi^*$ when the policy of node $j \in \{j | j \in subtree(N_i), \forall N_i \in \mathbb{C}\}$.
%\end{theorem}
